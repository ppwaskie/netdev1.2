% This file is isea.tex.  It contains the formatting instructions for and acts as a template for submissions to ISEA 2015.  It is based on the ICCC  formats and instructions.  It uses the files isea.sty, isea.bst and isea.bib, the first two of which also borrow from AAAI IJCAI formats and instructions.
% Modified from ICCC.tex by B. Bogart

\documentclass[letterpaper]{article}
\usepackage{isea}
\usepackage[pdftex]{graphicx}
\usepackage{times}
\usepackage{helvet}
\usepackage{courier}
\usepackage[numbers]{natbib}
\pdfinfo{
/Title (Scaling with multiple network namespaces in a single application)
/Author (PJ Waskiewicz)}
% The file isea.sty is the style file for ISEA 2015 proceedings.
%
\title{Scaling With Multiple Network Namespaces in a Single Application}
\author{PJ Waskiewicz\\
Principal Engineer at NetApp\\
Portland, OR, USA\\
pj.waskiewicz@netapp.com\\
\newline
\newline
}
\setcounter{secnumdepth}{0}

\begin{document} 
\maketitle
\begin{abstract}
Namespaces and containers seem to be the rage today, but it’s not typical for large applications to use them directly. Rather, applications rely on frameworks such as LXC and/or Docker to manage the containers they run in.

This paper will focus on how a large application can utilize the network namespace framework, and how it can use large numbers of namespaces to partition up the underlying network infrastructure within the application. An overview of parts of the application architecture using the namespaces will be covered, showing the use cases driving the need for namespaces. Lessons learned around scalability and performance bottlenecks in the kernel will be shared.

Ultimately this paper will propose further improvements to the namespace framework for better programmatic management of namespaces within the kernel from userspace, as well as attention to increased scalability and efficiency of networking within the namespaces.
\end{abstract}

\section{Keywords}

networking, kernel, containers, namespaces, scaling, cloud

\section{Introduction}
The ongoing evolution of the datacenter towards cloud-based infrastructure continues to present interesting challenges to existing applications. These existing solutions, e.g. storage appliances, work to serve multiple tenants within a computing domain. However, when the infrastructure around these solutions evolves into a cloud-based, partitioned environment, these solutions must also evolve.

When an application has core logic that needs to span multiple, separate network environments, it must become container/network-namespace aware. The Linux kernel exposes an API to create and manage network namespaces within an application. This paper will focus on this API for the following:
\begin{itemize}
\item How to create and manage the lifecycle of a network namespace within a complex application
\item How to track and process multiple network connections across multiple network namespaces in an efficient and scalable manner
\item What limitations exist in this API that makes lifecycle management a challenge, along with proposals on how to improve the API for better lifecycle management
\item Scalability issues encountered, how these were addressed, and proposals around scalability testing to prevent regressions
\end{itemize}
\section{Style and Format}
Templates that implement these instructions can be retrieved electronically at {\small \tt http://isea2015.org}

\subsection{Layout}

Print manuscripts two columns to a page, in the manner in which these instructions are printed. The exact dimensions for pages are:
\begin{itemize}
\item left and right margins: 0.75''
\item column width: 3.31''
\item gap between columns: 0.38''
\item top margin—first page: 1.25''
\item top margin—other pages: 0.75''
\item bottom margin: 1.25''
\end{itemize}

\subsection{Format of Electronic Manuscript}

For the production of the electronic manuscript, you must use {\em Adobe's Portable Document Format} (PDF). Additionally, you must specify the American {\em letter} format (corresponding to 8-1/2'' x 11'') when formatting the paper.

\subsection{Blind Review}

All papers will be reviewed in a single blind manner.  You are at liberty to include your affiliation and cite your papers in a natural manner, and you are also at liberty to anonymize the text if you so desire, in which case, keeping your identity secret is your responsibility.

\subsection{Title and Author Information}

Center the title on the entire width of the page in a 16-point bold font. Below it, center the author name(s) in a 12-point bold font, and then center the address(es) in a 9-point regular font. Credit to a sponsoring agency can appear in the Acknowledgment Section described below.

\begin{figure}[h]
\includegraphics[width=3.31in]{standard-app-overview.png}
\caption{High-level architecture of application.}
\end{figure}

\subsection{Abstract}

The title ``Abstract'' should be 10 point, bold type, centered at the beginning of the left column. The body of the abstract summarizing the thesis and conclusion of the paper in no more than 200 words should be 9 point, justified, regular type.

\subsection{Text}

The main body of the text immediately follows the abstract. Use 10-point type in {\em Times New Roman} font.

Indent when starting a new paragraph, except after major headings. 

\subsection{Headings and Sections}

When necessary, headings should be used to separate major sections of your paper. (These instructions use many headings to demonstrate their appearance; your paper should have fewer headings.

\subsubsection{Section Headings}

Print section headings centered, in 12-point bold type in the style shown in these instructions. Your body text should be 10 point, justified, single space. Do not number sections.

\subsubsection{Subsection Headings}

Print subsection headings left justified, in 11-point bold type and mixed case (nouns, pronouns, and verbs are capitalized). They should be flush left. Your text should be 10 point, justified, single space. Do not number subsections.

\subsubsection{Subsubsection Headings}

Print subsubsection headings inline in 10-point bold type. Do not number subsubsections.

\subsubsection{Special Sections}

You may include an unnumbered acknowledgments section, including acknowledgments of help from colleagues, financial support, and permission to publish.

The references section is headed ``References,'' printed in the same style as a section heading. A sample list of references is given at the end of these instructions.  Note the various examples for books, proceedings, multiple authors, etc. 

\subsection{Footnotes}

If footnotes are necessary, place them at the bottom of the page in 9-point font. Refer to them with superscript numbers.\footnote{This is how your footnotes should appear.} Separate them from the text by a short horizontal line. 

\subsection{Itemized Lists}

Itemized lists shall use the en-dash as item. Let’s take the case of URL, automatic links and punctuation as an example:
Turn off the automatic linking feature for URLs in Word.

Quotations: For direct quotations remember to use ``double inverted commas.'' Quotations must be carefully transcribed and accurate. 

Periods and commas go inside quotation marks. This applies to ``double inverted commas,'' as well as single `inverted commas,' and to the use of a full stop as in the ``following example.'' 

Parenthesis: When an entire sentence is enclosed in parentheses, the punctuation mark belongs inside the closing parenthesis as in this example: applying this may be difficult at times. (We think it is important.)

\begin{itemize}
\item The punctuation mark belongs outside the closing parenthesis if the brackets are within the sentence as in this example: applying this may be difficult at times, but good results are guaranteed (and this is important).
\item Use en dashes with spaces -- like this -- to set off phrases. En dashes are moreover placed between digits to indicate a range (1--10 October; pp. 25--30). You can type an en dash with ALT + 0150 (in the numeric keypad) in Windows, or OPTION + HYPHEN in Mac.
\end{itemize}

\subsection{Quotations and Extracts}
Indent long quotations and extracts by 10 points at left margins.

\section{Acknowledgments}
We would like to acknowledge the engineers who contributed to this namespace-driven design: Joe Roback, Carl Seelye, Tom Distler, Jared Cantwell, and Marshall McMullen.

We would also like to acknowledge the Netdev1.2 selection committee for inviting us to submit and present this paper.

\begin{figure*}
\includegraphics[width=\textwidth]{two-column-figure.png}
\caption{Example of a double-column figure with caption. \copyright Respect Copyright.}
\end{figure*}

\section{Bibliography}
The title ``Bibliography'' should be 12 point, bold style, centered. Using 9 point, regular type, list your bibliography in alphabetical order by family name, after the references. The difference between a reference list and a bibliography is that in your references, you list all the sources you directly referred to in the body of your writing in numerical order, whereas a bibliography includes an alphabetical listing of all those authors and sources that you have consulted while writing your essay. Use the same format as for the references otherwise. Those using Latex will follow the usual cite command format \cite{boden92}.

\section{Author Biography}
PJ Waskiewicz is a Principal Software Engineer at NetApp in the SolidFire division. Prior to SolidFire/NetApp, PJ worked for many years as a network kernel engineer and device driver developer in the Networking Division of Intel.  He also worked on the x86 kernel tree, enabling advanced features in the Broadwell and Skylake microarchitectures.

\bibliographystyle{isea}
\bibliography{isea}


\end{document}
